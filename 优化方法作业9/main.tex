\documentclass{homework}
\usepackage{ctex, bm}
\usepackage{makecell}
\usepackage[ruled,vlined]{algorithm2e}
\usepackage{booktabs}
\usepackage{multirow}
\usepackage{siunitx}
\usepackage{graphicx}
% \usepackage{subcaption}
\usepackage{caption}
\newcommand{\bfx}{\mathbf{x}}
\newcommand{\bfg}{\mathbf{g}}
\newcommand{\bfH}{\mathbf{H}}
\newcommand{\bfd}{\mathbf{d}}
\author{李健宁}
\class{机器学习中的优化问题}
\date{\today}
\title{Homework 9}
% \address{Bayt El-Hikmah}

\graphicspath{{./media/}}

\begin{document} \maketitle

\question

\begin{sol}
\subsection*{梯度与 Hessian}
\[
\nabla f(x)
=\begin{pmatrix}
\frac{\partial f}{\partial x_1}\\[6pt]
\frac{\partial f}{\partial x_2}
\end{pmatrix}
=\begin{pmatrix}
x_1^3 - x_2 + 1\\[4pt]
x_2 - x_1 - 1
\end{pmatrix}, 
\qquad
\nabla^2 f(x)
=\begin{pmatrix}
3x_1^2 & -1\\
-1 & 1
\end{pmatrix}.
\]

\subsection*{DFP 算法迭代}
取 $H_0=I_2$,分别对两组初始值做迭代:
\begin{table}[ht]
\centering
\caption{DFP 算法在不同初始点的迭代过程比较}
\label{tab:dfp_compare}
\begin{tabular}{c c c c c c | c c c c c c}
\toprule
\multicolumn{6}{c|}{初始点 $(0,0)^T$} & \multicolumn{6}{c}{初始点 $(1.5,1.0)^T$} \\
\midrule
$k$ & $x_1^{(k)}$ & $x_2^{(k)}$ & $f(x^{(k)})$ & $\|\nabla f(x^{(k)})\|$ & $\alpha_k$ 
& $k$ & $x_1^{(k)}$ & $x_2^{(k)}$ & $f(x^{(k)})$ & $\|\nabla f(x^{(k)})\|$ & $\alpha_k$ \\
\midrule
0 &  0.000000 &  0.000000 &  0.000000 & 1.414214 & 0.596072 
& 0 &  1.500000 & 1.000000 &  0.765625 & 3.693322 & 0.254111 \\
1 & -0.596072 &  0.596072 & -0.627631 & 0.271732 & 1.586866 
& 1 &  0.642375 & 1.381167 & -0.629639 & 0.285845 & 2.235703 \\
2 & -0.946621 &  0.358522 & -0.700745 & 0.368605 & 2.422839 
& 2 &  0.855450 & 1.981643 & -0.724054 & 0.377354 & 1.747929 \\
3 & -1.050298 & -0.007492 & -0.746425 & 0.157065 & 0.644399 
& 3 &  1.012417 & 2.078697 & -0.747647 & 0.077927 & 1.070668 \\
4 & -0.999043 &  0.003383 & -0.749996 & 0.002480 & 0.999313 
& 4 &  0.997143 & 1.998481 & -0.749991 & 0.007154 & 0.834262 \\
5 & -0.999946 &  0.000014 & -0.750000 & 0.000152 & 1.059407 
& 5 &  0.999995 & 1.999972 & -0.750000 & 0.000025 & 1.001478 \\
6 & -1.000000 & -0.000000 & -0.750000 & 0.000000 & —        
& 6 &  1.000000 & 2.000000 & -0.750000 & 0.000000 & —        \\
\bottomrule
\end{tabular}
\end{table}

两组不同初始点分别落入了函数的两个局部最小点$(-1,0)$和$(1,2)$。故 DFP 算法并未收敛到同一个点,而是陷入了不同的局部最优解。
\end{sol}

\question

\begin{sol}

两种算法都顺利求出了最优解$x = (3,9,84)^\top$,且迭代过程的所有$H$都是正定的。图\ref{2-1}展示了两种算法下梯度范数随迭代次数的变化。具体的算法代码请见附件\texttt{HW9-code.ipynb}。

\img<2-1>[0.5]{两种算法下梯度范数随迭代次数的变化}{2-1.png}

\end{sol}

\question

\begin{sol}

使用L-BFGS算法和Wolfe条件。具体的参数选择上,$f(x)$的参数$\alpha = 100$,L-BFGS算法的$m\in [1,3,5,10,20,30]$,停止条件$\nabla f(x) \le 10^{-6}$,最大迭代次数$1000$次,Wolfe条件中初始$\alpha = 1.0, c_1 = 10^{-4},c_2=0.9$,采用二分法最多迭代$20$次,寻找满足条件的$\alpha_k$。

最终结果如表\ref{3-1},$m$的增大增加了对内存的占用,但也加速了迭代。当$m$超过实际迭代次数时,L-BFGS算法实际上就是BFGS算法,因此达到稳定。
\begin{table}[h]
\caption{不同$m$对L-BFGS算法的影响}\label{3-1}
\begin{tabular}{cccc}
\toprule
\textbf{m} & \textbf{迭代次数} & \textbf{最终函数值} & \textbf{时间 (s)} \\
\midrule
1  & 70 & $3.87 \times 10^{-15}$ & 0.18 \\
3  & 27 & $9.04 \times 10^{-17}$ & 0.11 \\
5  & 26 & $2.51 \times 10^{-20}$ & 0.06 \\
10 & 25 & $1.92 \times 10^{-16}$ & 0.06 \\
20 & 25 & $8.55 \times 10^{-17}$ & 0.06 \\
30 & 25 & $8.55 \times 10^{-17}$ & 0.06 \\
\bottomrule
\end{tabular}
\end{table}
\end{sol}

\question 

\begin{sol}
    
\subsection*{Lipschitz gradient majorant}

设 $f(x)=\frac12\|Ax-b\|_2^2$,其梯度 Lipschitz 常数为 $L=\lambda{\max}(A^TA)$,则有:
\[f(x) \leq f(x^{(k)}) + \nabla f(x^{(k)})^T(x - x^{(k)}) + \frac{L}{2} \|x - x^{(k)}\|_2^2\]
也即$$\frac12\|Ax-b\|_2^2+\lambda \|x\|_1\le f(x^{(k)}) + \nabla f(x^{(k)})^T(x - x^{(k)}) + \frac{L}{2} \|x - x^{(k)}\|_2^2 +\lambda \|x\|_1:=g_k(x)$$
子问题$\min g_k(x)$解析解为\[x^{(k+1)} = \mathrm{soft}_\frac{\lambda}{L}( x^{(k)} - \frac{1}{L} A^T(Ax^{(k)} - b))\]

其中 $\mathrm{soft}_\tau(u)=\mathrm{sign}(u)\max\{|u|-\tau,0\}$。迭代$1000$次,得到结果。


\subsection*{Variational majorant function}

利用恒等式
\[|x_i|=\min_{d_i>0}\;\frac12\bigl(d_i x_i^2 + d_i^{-1}\bigr),\]
有
$$\frac12\|Ax-b\|_2^2+\lambda \|x\|_1\le \frac12\|Ax-b\|_2^2 + \lambda [\frac12(x^{\top}\mathrm{D}x + \mathrm{1}^{\top}\mathrm{D}^{-1}\mathrm{1})]:=  h_k(x,d)$$

易求得
\[ \arg \min_d h_k(x^{(k)},d) = \mathrm{diag}(\frac1{|x_i^{(k)}| }).\]
为保证数值稳定,取$d_i^{(k)} = \frac1{|x_i^{(k)}| + \varepsilon}$。那么

\[
g_k(x) = \frac12\|Ax-b\|_2^2
+\frac{\lambda}{2}\sum_i d_i^{(k)}x_i^2 +C,\ 
x^{(k+1)} 
=\arg\min_x\;g_k(x),
\]
($C$为与 $x$ 无关常数),则$x^{(k+1)}$是线性方程组
\[
\bigl(A^TA + \lambda\,\mathrm{diag}(d^{(k)})\bigr)\,x} = A^Tb
\]
的解。
\subsection*{数据实验}

数据实验中,我们取$A$为$m\times n$的取值在$[0,1]$的随机矩阵,生成仅有前10个分量不为0的随机稀疏解向量$x_{true}$,根据$x_{true}$,计算$b = Ax_{true} + 0.01 s$,$s$为随机生成的每个分量取值在$[0,1]$的$m$维向量。

取$m=100,n=200,\lambda = 0.1, \varepsilon = 10^{-6}$,迭代$5000$次,得到的结果如图\ref{4-1}和表\ref{4-2}。需要说明的是$x_{true}$并非原问题的真实最优解,两种方法求出的最优解与$x_{true}$的差的L2范数和我们为$b$引入的随机扰动的尺度(每个分量方差为$0.01$)相吻合。

\img<4-1>[0.5]{两种算法下L2范数随迭代次数的变化}{4-1.png}

\begin{table}[h]
\caption{两种算法下的最终函数值和L2范数}\label{4-2}
\begin{tabular}{ccc}
\toprule
\textbf{Majorant} & \makecell{\textbf{最终解$x$与$x_{true}$的L2范数}\\$\|x-x_{true}\|_2$} & \textbf{最终函数值} \\
\midrule
Lipschitz $x=x_L$  & $7.2321 \times 10^{-3}$ & $6.6589 \times 10^{-1}$\\
Variational $x=x_V$  & $7.2209 \times 10^{-3}$ & $6.6590 \times 10^{-1}$ \\
相对误差$\|x_L-x_V\|_2$ & $1.1226 \times 10^{-4}$ & \\
\bottomrule
\end{tabular}
\end{table}

\end{sol}



% citations
% \bibliographystyle{plain}
% \bibliography{citations}

\end{document}
